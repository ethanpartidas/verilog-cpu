\documentclass{article}
\usepackage{graphicx} % Required for inserting images
\usepackage{multirow}
\usepackage[a4paper, total={6in, 9in}]{geometry}
\usepackage{parskip}
\usepackage{tikz}
\usetikzlibrary{shapes.geometric, arrows}
\usepackage{listings}
\lstset{
basicstyle=\small\ttfamily,
columns=flexible,
breaklines=true
}

\title{HDL Final Project Report}
\author{Ethan Partidas}
\date{April 2023}

\begin{document}

\maketitle

\section{Processor Specifications}

This processor is an 8 bit processor with a single-cycle datapath written in Verilog. Each register stores an 8 bit value, and all operations are 8 bit. Instructions are 16 bit, and therefore are accessed by two consecutive locations in memory.

The processor has 3 inputs and 4 outputs:

\begin{itemize}
    \item clock: clock signal
    \item reset: indicates that the internal CPU variables should be reset to 0.
    \item memory: gives the CPU access to the main memory
    \item write\_address: where the CPU wants to write to in memory
    \item write\_value: the value the CPU wants to write at write\_address
    \item write: whether the CPU wants to write to memory
    \item halt: whether the CPU has finished execution
\end{itemize}

The ISA of the CPU has 16 instructions. Each instruction is split into 3 or 4 fields with 4 or 8 bits per field. The following is a summary of those instructions.

\centering

\vspace{1em}
\begin{tabular}{|c|c|c|c|c|}
    \hline
    Instruction & 15:12 & 11:8 & 7:4 & 3:0 \\
    \hline
    HALT & 0 & X & X & X \\
    LOAD & 1 & RD & RS & X \\
    STORE & 2 & RD & RS & X \\
    LI & 3 & RD & \multicolumn{2}{c|}{IM} \\
    ADD & 4 & RD & RS1 & RS2 \\
    SUB & 5 & RD & RS1 & RS2 \\
    MUL & 6 & RD & RS1 & RS2 \\
    DIV & 7 & RD & RS1 & RS2 \\
    NOT & 8 & RD & RS & X \\
    AND & 9 & RD & RS1 & RS2 \\
    OR & A & RD & RS1 & RS2 \\
    XOR & B & RD & RS1 & RS2 \\
    JUMP & C & X & \multicolumn{2}{c|}{IM} \\
    BEQ & D & RD & RS1 & RS2 \\
    BGT & E & RD & RS1 & RS2 \\
    BLT & F & RD & RS1 & RS2 \\
    \hline
\end{tabular}
\vspace{1em}

\raggedright

They all do what you expect. For an exact specification of what each instruction does, see the HDL Code section.

\section{Block Diagram}

\centering

\tikzstyle{box} = [rectangle, minimum width=3cm, minimum height=3cm, text centered, draw=black]
\tikzstyle{smallbox} = [rectangle, minimum width=1cm, minimum height=1cm, text centered, draw=black]

\begin{tikzpicture}[node distance=2cm]
\node (cpu) [box] {CPU};
\node (memory) [box, right of=cpu, xshift=3cm] {Memory};
\node (registers) [smallbox, yshift=-0.8cm] {Registers};
\draw [] (cpu) -- (memory);
\end{tikzpicture}

\raggedright

The registers are contained within the CPU module, and the memory exists as an array of bytes outside of the CPU.

\section{Test Program}

The program we will use to test the CPU will generate 10 Fibonacci numbers. This involves using two registers X and Y to store pairs of consecutive Fibonacci numbers, and adding them repeatedly to get the desired output. The output is written into memory, which the test bench will then print out.

\subsection{Registers}

\centering

\vspace{1em}
\begin{tabular}{|c|c|}
    \hline
    Register & Value \\
    \hline
    0 & Counter \\
    1 & 0A \\
    2 & Register X \\
    3 & Register Y \\
    4 & 01 \\
    5 & Branch Destination \\
    6 & Write Address \\
    7 & Temp \\
    8 & 00 \\
    \hline
\end{tabular}

\raggedright

\subsection{Assembly Code}

\centering

\vspace{1em}
\begin{tabular}{|c|c|c|}
    \hline
    Address & Instruction & Byte Code \\
    \hline
    00 & LI 1,0A & 310A \\
    02 & LI 2,01 & 3201 \\
    04 & LI 3,01 & 3301 \\
    06 & LI 4,01 & 3401 \\
    08 & LI 5,1E & 351E \\
    0A & LI 6,80 & 3680 \\
    0C & BEQ 5,0,1 & D501 \\
    0E & STORE 6,2 & 2620 \\
    10 & ADD 7,2,3 & 4723 \\
    12 & OR 2,3,8 & A238 \\
    14 & OR 3,7,8 & A378 \\
    16 & ADD 0,0,4 & 4004 \\
    18 & ADD 6,6,4 & 4664 \\
    1A & JUMP 0C & C00C \\
    1E & HALT & 0000 \\
    \hline
\end{tabular}

\raggedright

\section{HDL Code}

\subsection{CPU}

\begin{lstlisting}
module xcpu (
  input clock,
  input reset,
  input reg [7:0] memory [0:255],
  output reg [7:0] write_address,
  output reg [7:0] write_value,
  output reg write,
  output reg halt
);

  reg [7:0] program_counter = 0;

  reg [7:0] register_file [0:15];

  reg [3:0] operation_code = 0;
  reg [3:0] destination_register = 0;
  reg [3:0] source_register_1 = 0;
  reg [3:0] source_register_2 = 0;
  reg [7:0] immediate = 0;

  integer i;

  always @(posedge clock) begin
    if (reset) begin
      write_address = 0;
      write_value = 0;
      write = 0;
      halt = 0;
      program_counter = 0;

      for (i = 0; i < 16; i = i + 1) begin
        register_file[i] = 0;
      end

    end else if (~halt) begin
      operation_code = memory[program_counter][7:4];
      destination_register = memory[program_counter][3:0];
      source_register_1 = memory[program_counter+1][7:4];
      source_register_2 = memory[program_counter+1][3:0];
      immediate = memory[program_counter+1];

      write = 0;

      case (operation_code)
        4'b0000: begin // HALT
          halt = 1;
        end
        4'b0001: begin // LOAD
          register_file[destination_register] = memory[register_file[source_register_1]];
        end
        4'b0010: begin // STORE
          write_address = register_file[destination_register];
          write_value = register_file[source_register_1];
          write = 1;
        end
        4'b0011: begin // LI
          register_file[destination_register] = immediate;
        end
        4'b0100: begin // ADD
          register_file[destination_register] = register_file[source_register_1] + register_file[source_register_2];
        end
        4'b0101: begin // SUB
          register_file[destination_register] = register_file[source_register_1] - register_file[source_register_2];
        end
        4'b0110: begin // MUL
          register_file[destination_register] = register_file[source_register_1] * register_file[source_register_2];
        end
        4'b0111: begin // DIV
          register_file[destination_register] = register_file[source_register_1] / register_file[source_register_2];
        end
        4'b1000: begin // NOT
          register_file[destination_register] = ~register_file[source_register_1];
        end
        4'b1001: begin // AND
          register_file[destination_register] = register_file[source_register_1] & register_file[source_register_2];
        end
        4'b1010: begin // OR
          register_file[destination_register] = register_file[source_register_1] | register_file[source_register_2];
        end
        4'b1011: begin // XOR
          register_file[destination_register] = register_file[source_register_1] ^ register_file[source_register_2];
        end
        4'b1100: begin // JUMP
          program_counter = immediate - 2;
        end
        4'b1101: begin // BEQ
          if (register_file[source_register_1] == register_file[source_register_2]) begin
            program_counter = register_file[destination_register] - 2;
          end
        end
        4'b1110: begin // BGT
          if (register_file[source_register_1] > register_file[source_register_2]) begin
            program_counter = register_file[destination_register] - 2;
          end
        end
        4'b1111: begin // BLT
          if (register_file[source_register_1] < register_file[source_register_2]) begin
            program_counter = register_file[destination_register] - 2;
          end
        end
      endcase
      program_counter = program_counter + 2;
    end
  end
endmodule
\end{lstlisting}

\subsection{Test Bench}

\begin{lstlisting}
// Testbench
module test;

  reg clock;
  reg reset;
  reg [7:0] memory [0:255];
  reg [7:0] write_address;
  reg [7:0] write_value;
  reg write;
  reg halt;

  integer i;

  // Instantiate design under test
  xcpu XCPU(
    .clock(clock),
    .reset(reset),
    .memory(memory),
    .write_address(write_address),
    .write_value(write_value),
    .write(write),
    .halt(halt)
  );

  initial begin
    // Dump waves
    $dumpfile("dump.vcd");
    $dumpvars(1);

    clock = 0;
    for (i = 0; i < 256; i = i + 1) begin
      memory[i] <= 0;
    end

    memory[0] <= 'h31;
    memory[1] <= 'h0A;
    memory[2] <= 'h32;
    memory[3] <= 'h01;
    memory[4] <= 'h33;
    memory[5] <= 'h01;
    memory[6] <= 'h34;
    memory[7] <= 'h01;
    memory[8] <= 'h35;
    memory[9] <= 'h1E;
    memory[10] <= 'h36;
    memory[11] <= 'h80;
    memory[12] <= 'hD5;
    memory[13] <= 'h01;
    memory[14] <= 'h26;
    memory[15] <= 'h20;
    memory[16] <= 'h47;
    memory[17] <= 'h23;
    memory[18] <= 'hA2;
    memory[19] <= 'h38;
    memory[20] <= 'hA3;
    memory[21] <= 'h78;
    memory[22] <= 'h40;
    memory[23] <= 'h04;
    memory[24] <= 'h46;
    memory[25] <= 'h64;
    memory[26] <= 'hC0;
    memory[27] <= 'h0C;

    reset = 1;
    cycle;
    reset = 0;

    while (~halt) begin
      cycle;
      if (write) begin
        memory[write_address] <= write_value;
      end
    end

    $display("Results.");
    for (i = 0; i < 10; i = i + 1) begin
      $display("%d", memory['h80+i]);
    end
  end

  task cycle;
    #1 clock = 1;
    #1 clock = 0;
  endtask

endmodule
\end{lstlisting}

\section{Simulation Results}

\subsection{Text Output}

\begin{lstlisting}
# KERNEL: Reset.
# KERNEL: Results.
# KERNEL:   1
# KERNEL:   1
# KERNEL:   2
# KERNEL:   3
# KERNEL:   5
# KERNEL:   8
# KERNEL:  13
# KERNEL:  21
# KERNEL:  34
# KERNEL:  55
# KERNEL: Simulation has finished. There are no more test vectors to simulate.
# VSIM: Simulation has finished.
Done
\end{lstlisting}

\subsection{Waveform Output}

\includegraphics[width=6in]{waveform.png}

\section{Conclusion}

I've always wondered what it would be like to design a CPU from scratch, and with Verilog, it is surprisingly easy! It does all the hard work of implementing the state machines, allowing the programmer to focus on designing their ISA.

\end{document}
